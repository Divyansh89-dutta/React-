immutable vs mutable
immutable :- 
1. Immutable objects are objects that cannot be changed once they are created.
2. Immutable objects are thread-safe.
mutable :-
1. Mutable objects are objects that can be changed once they are created.
2. Mutable objects are not thread-safe.

primitive and refernce 
primitive :-
1. Primitive types are the basic data types in a programming language.
2. Primitive types are immutable.
reference :-
1. Reference types are the complex data types in a programming language.
2. Reference types are mutable.

array objects destructing import and export 
array objects :-
1. Array objects are mutable.
2. Array objects are reference types.
destructing :-
1. Destructuring is a feature in JavaScript that allows you to unpack values from arrays, objects
2. Destructuring is used to assign values to variables from an array or object.
import and export :-
1. Import and export are used to share code between modules.
2. Import is used to bring variables, functions, and classes from other modules into the current module
3. Export is used to make variables, functions, and classes available for import in other modules.

map filter arrow fncs (implicit return) spread opration

React js mein aapko ek state naam ki cheej milege us bande ko app mutable nahi krskte matlab ki directly uski value nahi hata yo jod sakte 

var state = [1,2,3,4];
% state.push(5); // state = [1,2,3,4,5]
 state.pop(); // state = [1,2,3,4]
 this is not allowed in react 
 correct way is 
 var state = [1,2,3,4];
 state = [1,2] is the correct way

aur ab jo padhai shuru hogi wo isi baare mein hogi ke state ko immutable way mein kasie update karein 

how to copy 
var state = [1,2,3,4,5];
var copy = [...state];
copy.pop()
is the correct way

now for object 
var state = {a:1,b:2,c:3,d:4,e:5};
var copy = {...state};
copy.a = 10;
is the correct way to copy object and update it.

var obj = {name: "divyansh, age:23}
const {age} = obj;
and how to call it is by using 
const {age} = obj;
and just type age where you want to apply the variable

var obj = {name: "string" , social:{
    facebook: {
        first: "haah",
        second: "divyansh"
    }
}}
const {second} = obj.social;
now use the second now you good to goo 

hum log component banaate hai , component matlab page ka hissa, navbar, sidebar, cart,landing,second, ab dikkat ye aati hai ki har hissa alag alag component hai aur component ko hum alag alag files mein rakhte hai,to inko last mein jodna bhi padta hai, jodne ke liye use hota hai import and export

navbar - export 
sidebar - export
cart - export
main - import navba, sidebar, cart 

how to export :- 
function navbar(){
    return <div>navbar</div>;
}
export default navbar;
export function navbar(){
    return <div>navbar</div>;
}

how to import :- 
import navbar from './navbar.js';
function main(){
    return <div>
    <navbar/>
    </div>;
    }
import {Cart, Abcd} from
'./cart.js';

arrow function :-
arrow function is a function which is defined with arrow operator (=>) and it is used to define
a function in a single line. It is also known as lambda function.
type of arrow function :- 
1. single line arrow function :-
const abcd ()=>{
    return "divyansh";
}
2. multi line arrow function :-
const ac = () =>{
    conslelog(val);
}
3. arrow function with parameters :-
const ac = () => 12;
consle.log("hey"+ac());
4. arrow function with return type :-
const abcd = () => ({name: 'abcd'});
console.log(abcd().name); // Output: "abcd"

Map filter :-

map() :- it is used to create a new array from an existing array. It is used to
perform some operation on each element of the array and return a new array.

filter() :- it is used to create a new array with all elements that pass the test implemented by
the provided function.

map filter - dono hi array pe chalta hai, aur dono kaa kaam hai array par kuchh perform karna ek naya array return karna 

const arr = [1, 2, 3, 4, 5];
const doubleArr = arr.map((x) => x * 2);
console.log(doubleArr); // Output: [2, 4, 6, 8,
10]
const evenArr = arr.filter((x) => x % 2 === 0);
console.log(evenArr); // Output: [2, 4]

var arr = [1,2,3,4,5];
map - har element par kuchh karo and naye array mein rakho 
foreach ke andar aaata hai function 
step one:- 
arr.map(var => {
    return var*2;
    console.log(var);
})
step two:-
const ans = arr.map(val => val*12);
map ke andar return krne ka wajah se hi elements naye array main place hote hai 
for example :- 
state ek array hai usmein numbers hai and aapko har number ko +1 karna hai 
const state = [1,2,3,4,5];
const ans = state.map(val => val+1);

ek array hai saare number ko ki 5 se bade ho unko 5 job dena and baaki numbers waise ke waise return karo 
const state = [1,2,3,4,5];
const ans = state.map(val => val>5 ? 5 : val);

map filter mein ek hai farak hai, map saare bande lautata hai matlab ki count kam nahi hoga , filter bando ko kam kar skta hai

ek array mein sabhi nums jo ki 5 se bade hai unmein 10 add karo , jab og array ka size kam naa hona ho waha map use hota hai ,jab og ka size karna ho to waha filter use hota hai 
const state = [1,2,3,4,5,6,7,8,
9,10];
const arr = arr.filter(elem = > ture);
const ans = state.map(val => val>5 ? val+10 : val);
const arr = arr.filter(elem = > elem>4);